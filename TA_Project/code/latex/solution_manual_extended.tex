\documentclass[11pt,a4paper]{article}
\usepackage[utf8]{inputenc}
\usepackage{amsmath,amssymb}
\usepackage{geometry}
\usepackage{booktabs}
\usepackage{enumitem}
\geometry{margin=1in}
\setlist[itemize]{leftmargin=1.4em}
\setlist[enumerate]{leftmargin=1.6em}

\title{UT-ECE Data Science -- Extended Final \\
\textbf{Comprehensive Solution Manual (TA Edition)}}
\author{Course Staff}
\date{Spring 2025}

\begin{document}
\maketitle

\section*{Scoring Philosophy}
Use evidence-based grading: correctness, rigor, reproducibility, and interpretation quality.
Prefer transparent assumptions and explicit limitations over overconfident claims.

\section*{Q1. Lifecycle and Problem Framing}
\textbf{High-quality answer includes:}
\begin{itemize}
    \item clear business target (e.g., early identification of migration propensity),
    \item operational metric (AUC, recall@k, calibration, fairness constraints),
    \item lifecycle phases: framing -> collection -> validation -> modeling -> deployment -> monitoring,
    \item risk register (leakage, drift, policy shift, proxy bias).
\end{itemize}

\section*{Q2. Python/EDA}
\textbf{Expected components:}
\begin{itemize}
    \item dtype audit, null profile, duplicate checks, range sanity checks,
    \item at least six meaningful visualizations with non-trivial interpretation,
    \item modular preprocessing function with unit tests.
\end{itemize}

\section*{Q3. Scientific Studies and Inference}
\textbf{Key grading points:}
\begin{itemize}
    \item distinguishes observational limits from causal claims,
    \item states assumptions for CI/hypothesis testing,
    \item interprets p-values and confidence intervals correctly.
\end{itemize}

\textbf{Example framing:}
\begin{itemize}
    \item Null: $H_0: \Delta\mu=0$ for migration propensity proxy between cohorts.
    \item Use two-sample test with variance assumptions checked.
\end{itemize}

\section*{Q4. Visualization and Storytelling}
\textbf{Strong solution:}
\begin{itemize}
    \item KPI definitions tied to stakeholder decisions,
    \item perceptual design rationale (position/length over area/color where possible),
    \item explicit warning about misleading axis truncation or inappropriate color scales.
\end{itemize}

\section*{Q5. SQL Advanced Querying}
\textbf{Canonical moving-average query pattern:}
\begin{verbatim}
WITH citation_velocity AS (
  SELECT UserID, Country_Origin, Year, Research_Citations,
         AVG(Research_Citations) OVER (
           PARTITION BY Country_Origin
           ORDER BY Year
           ROWS BETWEEN 2 PRECEDING AND CURRENT ROW
         ) AS moving_avg_citations
  FROM Professionals_Data
)
SELECT *, DENSE_RANK() OVER (
  PARTITION BY Country_Origin ORDER BY moving_avg_citations DESC
) AS country_rank
FROM citation_velocity;
\end{verbatim}

\textbf{Additional SQL expectations:}
\begin{itemize}
    \item percentile bucketing (e.g., NTILE or percentile window),
    \item cohort/transition query via CTE.
\end{itemize}

\section*{Q6. Leakage and Big-Data Architecture}
\textbf{Leakage decisions:}
\begin{itemize}
    \item \texttt{Visa\_Approval\_Date}: direct leakage (post-outcome),
    \item \texttt{Last\_Login\_Region}: potential temporal leakage,
    \item \texttt{Passport\_Renewal\_Status}: possible temporal proxy leakage,
    \item \texttt{Years\_Since\_Degree}: acceptable if computed pre-inference.
\end{itemize}

\textbf{Architecture answer (acceptable):}
Bronze/Silver/Gold tables, feature store with point-in-time joins, online/offline feature parity, periodic drift checks.

\section*{Q7. Regression and Elastic Net}
For
\[
J(\theta)=\frac{1}{2m}\sum_{i=1}^m (h_\theta(x^{(i)})-y^{(i)})^2 + \lambda_1\sum_j|\theta_j| + \frac{\lambda_2}{2}\sum_j\theta_j^2,
\]
\[
\nabla_{\theta_j}J = \frac{1}{m}\sum_{i=1}^m(h_\theta(x^{(i)})-y^{(i)})x_j^{(i)} + \lambda_1\partial|\theta_j| + \lambda_2\theta_j,
\]
\[
\partial|\theta_j|=
\begin{cases}
+1 & \theta_j>0\\
-1 & \theta_j<0\\
[-1,1] & \theta_j=0
\end{cases}
\]

\section*{Q8. Optimization}
\textbf{Ravine intuition:}
steep curvature in one axis and shallow curvature in another causes SGD oscillation.

\textbf{Momentum:}
\[
v_t=\beta v_{t-1}+\eta g_t,\quad \theta_{t+1}=\theta_t-v_t
\]
Damps sign-flipping gradients and accelerates consistent directions.

\textbf{Adam:}
first and second moments with bias correction, parameter-wise scaling.

\section*{Q9. Model Family Comparison}
\textbf{Minimum expected protocol:}
\begin{itemize}
    \item fixed train/validation/test split with stratification,
    \item CV and hyperparameter tuning for each model family,
    \item metric table with at least AUC, F1, calibration/error analysis,
    \item interpretability discussion.
\end{itemize}

\section*{Q10. Dimensionality Reduction}
\textbf{PCA explained variance ratio:}
\[
\mathrm{EVR}_k = \frac{\lambda_k}{\sum_i\lambda_i}
\]
where $\lambda_k$ is variance captured by component $k$.

\section*{Q11. Clustering}
\textbf{K-Means elbow rationale:}
WCSS decreases monotonically with $K$, but marginal gain diminishes.

\textbf{Density alternative:}
DBSCAN robustness to non-spherical clusters and noise points.

\section*{Q12. Neural Networks and Sequence Models}
\textbf{Expected answer characteristics:}
\begin{itemize}
    \item clear architecture choice and training setup,
    \item baseline comparison against classical model,
    \item overfitting control (dropout/early stopping/regularization).
\end{itemize}

\section*{Q13. LMs and LLM Agents}
\textbf{Strong answer includes:}
\begin{itemize}
    \item agent workflow (plan -> retrieve -> reason -> verify),
    \item evaluation: faithfulness, hallucination, safety,
    \item governance boundaries and fallback logic.
\end{itemize}

\section*{Q14. Ethics and Fairness}
\textbf{Expected:}
\begin{itemize}
    \item subgroup metrics (e.g., by country/education),
    \item recognition of historical policy bias and proxy discrimination,
    \item mitigation and human override policy.
\end{itemize}

\section*{Capstone}
\textbf{Minimum complete capstone output:}
\begin{enumerate}
    \item leakage-safe preprocessing,
    \item best model with validated metrics,
    \item SHAP local explanation for high-citation no-migration case,
    \item global importance plot,
    \item fairness slice and deployment recommendation.
\end{enumerate}

\section*{Rubric Notes for TAs}
\begin{itemize}
    \item Deduct for hidden leakage or unjustified assumptions.
    \item Deduct for non-reproducible code.
    \item Reward honest limitations and rigorous diagnostics.
\end{itemize}

\end{document}
