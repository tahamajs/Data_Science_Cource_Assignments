\documentclass[11pt,a4paper]{article}
\usepackage{geometry}
\usepackage{booktabs}
\usepackage{longtable}
\usepackage{amsmath,amssymb}
\usepackage{enumitem}
\usepackage{tcolorbox}
\usepackage{array}
\usepackage{hyperref}
\usepackage{xepersian}

\settextfont{Times New Roman}
\setlatintextfont{Times New Roman}
\geometry{margin=1in}
\setlist[itemize]{leftmargin=1.8em}
\setlist[enumerate]{leftmargin=1.8em}
\hypersetup{colorlinks=true,linkcolor=blue,urlcolor=blue}

\title{دانشگاه تهران -- دانشکده مهندسی برق و کامپیوتر\\
\textbf{درس علم داده: تمرین نهایی جامع (نسخه حرفه‌ای توسعه‌یافته)}}
\author{تیم درس علم داده -- بهار ۱۴۰۴}
\date{ویرایش ۲.۰}

\begin{document}
\maketitle

\begin{tcolorbox}
\textbf{عنوان پروژه:} تحلیل مهاجرت جهانی استعدادهای فنی با رویکرد داده‌محور\\
\textbf{داده اصلی:} \lr{GlobalTechTalent\_50k.csv} (۵۰٬۰۰۰ رکورد)\\
\textbf{مسئله اصلی:} پیش‌بینی \lr{Migration\_Status} (۱=مهاجرت، ۰=عدم مهاجرت)
\end{tcolorbox}

\section*{۱) هدف آموزشی و فلسفه ارزیابی}
این تمرین نهایی برای سنجش هم‌زمان \textbf{درک ریاضی، پیاده‌سازی مهندسی، تحلیل انتقادی، و گزارش‌نویسی حرفه‌ای} طراحی شده است.
انتظار می‌رود دانشجو در این پروژه نشان دهد که می‌تواند:
\begin{itemize}
    \item مسئله واقعی را به یک مسئله داده‌محور قابل ارزیابی تبدیل کند؛
    \item از داده خام تا استقرار پیشنهادی، یک خط لوله کامل و بازتولیدپذیر بسازد؛
    \item از مدل‌های ساده تا مدل‌های غیرخطی/پیشرفته را با استدلال علمی مقایسه کند؛
    \item با روش‌های \lr{XAI} و تحلیل عدالت، خروجی مدل را قابل حسابرسی نماید.
\end{itemize}

\section*{۲) اقلام تحویلی الزامی}
\begin{enumerate}
    \item \textbf{نوت‌بوک اصلی پروژه} با ساختار \lr{Q1} تا \lr{Q20} و بخش \lr{Capstone}.
    \item \textbf{گزارش نهایی} (حداکثر ۲۵ صفحه بدون پیوست) شامل شکل‌ها، جدول‌ها، تفسیرها و محدودیت‌ها.
    \item \textbf{بسته کد} شامل اسکریپت‌های قابل اجرا، فایل وابستگی‌ها، و راهنمای اجرای کامل.
    \item \textbf{پاسخ‌نامه تشریحی} برای همه سوالات (فرمول، استدلال، خروجی).
    \item \textbf{خلاصه مدیریتی} ۱ تا ۲ صفحه برای مخاطب غیرتخصصی.
\end{enumerate}

\textbf{حداقل استاندارد فنی:}
\begin{itemize}
    \item ثبت \lr{random seed} در تمام بخش‌های تصادفی.
    \item تفکیک شفاف \lr{train/validation/test}.
    \item کنترل صریح نشت داده و نشت زمانی.
    \item ثبت نسخه کتابخانه‌ها و محیط اجرا.
    \item توان بازتولید نتایج صرفاً با اجرای دستورات اعلام‌شده.
\end{itemize}

\section*{۳) ساختار نمره‌دهی (۲۶۰ نمره)}
\begin{center}
\begin{tabular}{@{}p{1.7cm}p{7.8cm}p{2cm}@{}}
\toprule
بلوک & حوزه مهارتی & نمره \\
\midrule
A & مبانی: چرخه عمر علم داده، عملیات پایتونی، EDA & ۲۰ \\
B & استنباط آماری، طراحی بصری و روایت داده & ۲۰ \\
C & SQL پیشرفته و مهندسی داده در مقیاس & ۲۵ \\
D & مدل‌سازی نظارت‌شده و بهینه‌سازی & ۴۵ \\
E & یادگیری بدون نظارت و کاهش بُعد & ۲۰ \\
F & یادگیری عمیق، NLP، مدل‌های زبانی/عامل‌ها & ۳۰ \\
G & عدالت، سوگیری، حاکمیت و استقرار مسئولانه & ۱۵ \\
H & کپستون یکپارچه (پیاده‌سازی + تبیین + گزارش) & ۲۵ \\
I & توسعه حرفه‌ای پایش تولید و تحلیل‌های پیشرفته (\lr{Q15-Q20}) & ۶۰ \\
J (Bonus) & افزونه‌های پژوهشی/تولیدی پیشرفته & +۲۰ \\
\midrule
\textbf{جمع بلوک‌های اصلی} &  & \textbf{۲۶۰} \\
\textbf{امتیاز تشویقی} &  & \textbf{+۲۰} \\
\bottomrule
\end{tabular}
\end{center}

\section*{۴) صورت سوال توسعه‌یافته}

\subsection*{بلوک A -- مبانی (۲۰ نمره)}

\subsubsection*{Q1) چرخه عمر علم داده و صورت‌بندی مسئله (۱۰ نمره)}
یک سند فنی کوتاه ارائه دهید که شامل موارد زیر باشد:
\begin{enumerate}
    \item تعریف مسئله، ذی‌نفعان، و تصمیم‌هایی که مدل قرار است پشتیبانی کند.
    \item تعریف \lr{KPI}‌های فنی (مثل \lr{AUC}، \lr{Recall@K}، \lr{Calibration}) و عملیاتی.
    \item فهرست ریسک‌ها (نشت داده، \lr{drift}، تغییر سیاست مهاجرت، خطای برچسب).
    \item برنامه پایش پس از استقرار (آستانه هشدار، فرکانس بازآموزی، مسیر ارجاع انسانی).
\end{enumerate}
\textbf{خروجی مورد انتظار:} نمودار چرخه عمر + یک جدول ریسک.

\subsubsection*{Q2) عملیات پایتونی، کنترل کیفیت و EDA (۱۰ نمره)}
\begin{enumerate}
    \item پروفایل‌گیری داده: نوع ستون‌ها، مقادیر گمشده، تکراری، بازه‌های نامعتبر.
    \item شناسایی پرت‌ها با حداقل دو روش (مثلاً \lr{IQR} و \lr{z-score}) و تحلیل پیامد.
    \item حداقل ۸ نمودار اکتشافی با تفسیر کاربردی.
    \item پیاده‌سازی یک تابع پیش‌پردازش ماژولار و قابل آزمون.
\end{enumerate}

\subsection*{بلوک B -- استنباط آماری و مصورسازی (۲۰ نمره)}

\subsubsection*{Q3) طراحی مطالعه و استنباط (۱۰ نمره)}
\begin{itemize}
    \item تفکیک مطالعه مشاهده‌ای از ادعای علّی برای این مسئله.
    \item ارائه حداقل یک بازه اطمینان معتبر و تفسیر دقیق آن.
    \item تعریف یک آزمون فرض با \lr{H0/H1}، سطح معنی‌داری، کنترل خطای نوع اول.
    \item بررسی اعتبار پیش‌فرض‌ها (نرمال‌بودن تقریبی/حجم نمونه/استقلال).
\end{itemize}

\subsubsection*{Q4) طراحی بصری و روایت برای تصمیم‌گیر (۱۰ نمره)}
\begin{itemize}
    \item طراحی یک بخش داشبوردی برای مخاطب غیرتخصصی.
    \item توجیه انتخاب رنگ، مقیاس، ترتیب، و \lr{annotation}ها.
    \item ارائه حداقل یک مثال نمودار گمراه‌کننده + نسخه اصلاح‌شده.
\end{itemize}

\subsection*{بلوک C -- SQL و مهندسی داده (۲۵ نمره)}

\subsubsection*{Q5) SQL پیشرفته (۱۵ نمره)}
سه کوئری مستقل ارائه دهید:
\begin{enumerate}
    \item میانگین متحرک ۳ساله \lr{Research\_Citations} با \lr{window frame}.
    \item رتبه‌بندی/دهک‌بندی کشورها یا کاربران با \lr{RANK/DENSE\_RANK/NTILE}.
    \item یک تحلیل \lr{cohort} با \lr{CTE} (ماندگاری یا انتقال وضعیت در زمان).
\end{enumerate}

\subsubsection*{Q6) نشت داده و معماری داده در مقیاس (۱۰ نمره)}
\begin{itemize}
    \item ویژگی‌های نشت‌دهنده را شناسایی و با منطق زمانی حذف کنید.
    \item معماری \lr{Bronze/Silver/Gold} یا معادل آن را برای داده آفلاین/آنلاین پیشنهاد دهید.
    \item راهکار \lr{train-serving consistency} با \lr{feature store} توضیح دهید.
\end{itemize}

\subsection*{بلوک D -- مدل‌سازی نظارت‌شده و بهینه‌سازی (۴۵ نمره)}

\subsubsection*{Q7) مدل‌های خطی/لجستیک + \lr{Elastic Net} (۱۵ نمره)}
\begin{enumerate}
    \item مدل پایه خطی/لجستیک را اجرا کنید.
    \item تابع هزینه \lr{Elastic Net} را بنویسید و گرادیان/زیرگرادیان را استخراج کنید.
    \item تفسیر ضریب، \lr{p-value}، بازه اطمینان، و پایداری ضرایب را ارائه دهید.
\end{enumerate}

\subsubsection*{Q8) تحلیل بهینه‌سازی در \lr{Ravine} (۱۰ نمره)}
\begin{itemize}
    \item رفتار \lr{SGD}، \lr{Momentum}، و \lr{Adam} را روی یک تابع دره‌ای مقایسه کنید.
    \item مسیر همگرایی و نوسان را رسم و تحلیل کنید.
    \item برای ناهمگنی مقیاس ویژگی‌ها پیشنهاد عملی بدهید.
\end{itemize}

\subsubsection*{Q9) مقایسه خانواده مدل‌های غیرخطی (۲۰ نمره)}
\begin{itemize}
    \item \lr{SVM} و \lr{KNN}
    \item \lr{Decision Tree} و \lr{Random Forest}
    \item یک مدل \lr{Boosting} (ترجیحاً \lr{XGBoost})
\end{itemize}
الزامات:
\begin{enumerate}
    \item پروتکل \lr{cross-validation} و تنظیم ابرپارامتر.
    \item جدول متریک‌ها (حداقل: \lr{Accuracy, ROC-AUC, F1, Precision, Recall}).
    \item تحلیل خطا (الگوهای اشتباه، ماتریس درهم‌ریختگی، حساسیت تصمیم).
\end{enumerate}

\subsection*{بلوک E -- بدون نظارت (۲۰ نمره)}

\subsubsection*{Q10) کاهش بُعد (۱۰ نمره)}
\begin{itemize}
    \item \lr{PCA}: محاسبه و تفسیر \lr{Explained Variance Ratio}.
    \item یک روش مکمل (\lr{Random Projection}/\lr{t-SNE}/\lr{UMAP}).
    \item تفسیر هندسی ابعاد نهفته.
\end{itemize}

\subsubsection*{Q11) خوشه‌بندی (۱۰ نمره)}
\begin{itemize}
    \item \lr{KMeans} + \lr{Elbow} + \lr{Silhouette}.
    \item \lr{DBSCAN} یا روش چگالی‌محور معادل.
    \item تحلیل پایداری خوشه‌ها و معنای کاربردی آن‌ها.
\end{itemize}

\subsection*{بلوک F -- یادگیری عمیق، \lr{NLP} و مدل‌های زبانی (۳۰ نمره)}

\subsubsection*{Q12) مدل عصبی جدولی و مدل توالی/متن (۱۵ نمره)}
\begin{itemize}
    \item یک \lr{MLP} روی داده جدولی آموزش دهید.
    \item یک مدل توالی/متنی (\lr{RNN/LSTM/GRU/CNN}) اجرا کنید.
    \item عملکرد را با \lr{baseline} کلاسیک مقایسه کنید.
\end{itemize}

\subsubsection*{Q13) مدل‌های زبانی و \lr{LLM Agent} (۱۵ نمره)}
جریان عامل‌محور طراحی کنید:
\begin{center}
\lr{plan -> retrieve -> reason -> verify}
\end{center}
و برای آن:
\begin{itemize}
    \item معیار ارزیابی صحت/وفاداری/ایمنی تعریف کنید.
    \item سیاست مهار \lr{hallucination} و محدودیت دسترسی ابزارها مشخص کنید.
\end{itemize}

\subsection*{بلوک G -- اخلاق، عدالت و حاکمیت (۱۵ نمره)}

\subsubsection*{Q14) عدالت، سوگیری، و استقرار مسئولانه (۱۵ نمره)}
\begin{itemize}
    \item تحلیل زیرگروهی بر اساس کشور/تحصیلات/سابقه.
    \item بررسی \lr{proxy discrimination} و سوگیری تاریخی سیاست‌ها.
    \item ارائه سیاست \lr{human-in-the-loop} و مسیر اعتراض/بازبینی.
\end{itemize}

\subsection*{بلوک H -- کپستون یکپارچه (۲۵ نمره)}
خروجی کپستون باید شامل موارد زیر باشد:
\begin{enumerate}
    \item خط لوله کامل داده تا پیش‌بینی با کنترل نشت.
    \item یک گزارش \lr{model card} شامل مفروضات، محدودیت‌ها، متریک‌ها.
    \item تبیین محلی و سراسری با \lr{SHAP}.
    \item جدول عدالت زیرگروهی + توصیه استقرار + مانیتورینگ.
\end{enumerate}

\subsection*{بلوک I -- توسعه حرفه‌ای پایش تولید و تحلیل پیشرفته (۶۰ نمره)}

\subsubsection*{Q15) کالیبراسیون احتمال و سیاست آستانه (۱۰ نمره)}
\begin{itemize}
    \item منحنی کالیبراسیون/قابلیت اطمینان را برای بهترین مدل رسم کنید.
    \item حداقل یک معیار کالیبراسیون (مثل \lr{Brier} یا \lr{ECE}) گزارش کنید.
    \item دو سیاست آستانه ارائه دهید:
    \begin{enumerate}
        \item آستانه بهینه بر اساس \lr{F1}
        \item آستانه بهینه بر اساس هزینه نامتقارن خطا (مثلاً \lr{FN} پرهزینه‌تر از \lr{FP})
    \end{enumerate}
\end{itemize}

\subsubsection*{Q16) تشخیص درفت و طراحی مانیتورینگ (۱۰ نمره)}
\begin{itemize}
    \item داده را به دو پنجره مرجع/جاری تقسیم کنید (ترجیحاً زمانی).
    \item برای ویژگی‌های عددی، \lr{PSI} محاسبه و رتبه‌بندی کنید.
    \item حداقل یک شاخص درفت دسته‌ای (مثل \lr{JS divergence}) گزارش کنید.
    \item سیاست هشدار/بحرانی و محرک بازآموزی را تعریف کنید.
\end{itemize}

\subsubsection*{Q17) تحلیل \lr{Counterfactual Recourse} (۱۰ نمره)}
\begin{itemize}
    \item برای پیش‌بینی‌های منفی نزدیک آستانه، کمینه تغییر لازم برای تغییر تصمیم را برآورد کنید.
    \item حداقل دو ویژگی عملیاتی قابل مداخله را بررسی کنید.
    \item نرخ موفقیت ریکورس و میانه مقدار مداخله برای هر ویژگی را گزارش کنید.
    \item درباره امکان‌پذیری/اخلاقی‌بودن این مداخلات بحث کنید.
\end{itemize}

\subsubsection*{Q18) اعتبارسنجی زمانی و افت عملکرد (۱۰ نمره)}
\begin{itemize}
    \item اعتبارسنجی غلطان زمانی اجرا کنید (در صورت نبود ستون زمانی معتبر، fallback مستند ارائه دهید).
    \item متریک‌های هر fold (حداقل \lr{AUC/F1}) را گزارش کنید.
    \item افت عملکرد نسبت به fold اول و ارتباط آن با شاخص درفت را تحلیل کنید.
\end{itemize}

\subsubsection*{Q19) کمی‌سازی عدم‌قطعیت و پوشش (۱۰ نمره)}
\begin{itemize}
    \item بازه یا نمره اطمینان مبتنی بر \lr{conformal}/کالیبراسیون پیاده‌سازی کنید.
    \item پوشش تجربی را در چند سطح اطمینان گزارش کنید.
    \item پهنای بازه و ریسک کم‌پوششی را تحلیل و تفسیر کنید.
\end{itemize}

\subsubsection*{Q20) آزمایش مداخله عدالت الگوریتمی (۱۰ نمره)}
\begin{itemize}
    \item مدل پایه و معیارهای عدالت زیرگروهی را محاسبه کنید.
    \item یک روش مداخله (مثلاً \lr{reweighing}) اجرا و قبل/بعد را مقایسه کنید.
    \item قید سیاستی صریح تعریف کنید (مثلاً سقف افت \lr{AUC}) و نتیجه قبولی/رد ارائه دهید.
\end{itemize}

\subsection*{بلوک J (Bonus تا +۲۰) -- افزونه‌های پیچیده}
\begin{itemize}
    \item \textbf{Causal DAG و شناسایی (۵)}: گراف، مجموعه تعدیل، محدودیت‌های علّی را صریح بیان کنید.
    \item \textbf{Conformal / عدم‌قطعیت (۵)}: بازه یا امتیاز اطمینان با پوشش تجربی گزارش‌شده.
    \item \textbf{اعتبارسنجی زمانی (۵)}: اسپلـیت زمانی در برابر تصادفی، تحلیل افت عملکرد و درفت.
    \item \textbf{سروینگ آنلاین/استریمینگ (۵)}: طرح ویژگی‌های تازه، SLA، نگهبان OOD/درفت و مسیر rollback.
\end{itemize}

\section*{۵) قالب‌های تحویل و استاندارد مستندسازی}
\begin{itemize}
    \item همه شکل‌ها باید عنوان، واحد، و \lr{caption} دقیق داشته باشند.
    \item تمام ادعاها باید با خروجی کد/جدول پشتیبانی شوند.
    \item کد باید ماژولار، دارای نام‌گذاری استاندارد، و قابل اجرا با دستور واحد باشد.
    \item گزارش باید شامل بخش «محدودیت‌ها» و «کارهای آینده» باشد.
\end{itemize}

\section*{۶) معیار کسر نمره}
\begin{itemize}
    \item وجود نشت داده بدون گزارش: تا ۵۰\% کسر در بخش مرتبط.
    \item نبود بازتولیدپذیری: تا ۳۰\% کسر در پروژه.
    \item تفسیر نادرست آماری (مثلاً برداشت اشتباه از \lr{p-value}): کسر ۲۰\% در سوال.
    \item فقدان تحلیل عدالت یا اخلاق: کسر کامل نمره بلوک \lr{G}.
\end{itemize}

\section*{۷) امتیاز تشویقی (تا +۱۰ نمره)}
\begin{itemize}
    \item توسعه تحلیل علّی با \lr{DAG} و بحث شناسایی.
    \item افزودن \lr{temporal validation} و تحلیل \lr{drift} بین سال‌ها.
    \item اضافه‌کردن \lr{conformal prediction} یا برآورد عدم‌قطعیت پیشرفته.
\end{itemize}

\section*{۸) اصول اخلاق علمی}
\begin{itemize}
    \item استفاده از منبع بیرونی بدون ارجاع، تخلف آموزشی محسوب می‌شود.
    \item نتایج ضعیف/منفی نیز باید صادقانه گزارش شوند.
    \item استفاده از مدل‌های زبانی باید شفاف و مستند باشد.
\end{itemize}

\begin{tcolorbox}
\textbf{جمع‌بندی:} این تمرین صرفاً «ساخت مدل» نیست؛ هدف، ارائه یک \textbf{سامانه تصمیم‌یار قابل دفاع} است که از نظر فنی، آماری، اخلاقی و مهندسی در سطح حرفه‌ای دانشگاهی قابل ارزیابی باشد.
\end{tcolorbox}

\end{document}
