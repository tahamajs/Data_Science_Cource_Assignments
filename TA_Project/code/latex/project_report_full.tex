\documentclass[conference]{IEEEtran}
\usepackage{amsmath,amssymb}
\usepackage{booktabs}
\usepackage{graphicx}
\usepackage{float}
\usepackage{hyperref}
\usepackage{url}
\hypersetup{hidelinks}
\IfFileExists{../solutions/latex_metrics.tex}{% Auto-generated by report_metrics_export.py
% metric_export_version=2.0
\newcommand{\MetricDataSplitStrategy}{{'base\_supervised': 'stratified\_train\_test\_80\_20', 'q18\_temporal': 'fallback\_userid\_proxy', 'q19\_uncertainty': 'train/calibration/test = 60/20/20', 'q20\_fairness': 'stratified\_train\_test\_80\_20'}}
\newcommand{\MetricExportVersion}{2.0}
\newcommand{\MetricQFifteenBestFOneThreshold}{0.2500}
\newcommand{\MetricQFifteenBrier}{0.2436}
\newcommand{\MetricQFifteenEce}{0.0327}
\newcommand{\MetricQSixteenTopFeature}{Visa\_Approval\_Date}
\newcommand{\MetricQSixteenTopPsi}{0.0013}
\newcommand{\MetricQSeventeenRecourseRate}{1.0000}
\newcommand{\MetricQEighteenAucDecay}{0.0478}
\newcommand{\MetricQEighteenMeanAuc}{0.5428}
\newcommand{\MetricQNineteenCovNinety}{0.9000}
\newcommand{\MetricQNineteenMaxUndercoverage}{0.0050}
\newcommand{\MetricQTwentyDpGapBase}{0.1550}
\newcommand{\MetricQTwentyDpGapMitigated}{0.0978}
\newcommand{\MetricQTwentyPolicyPass}{true}
\newcommand{\MetricQSixAccuracy}{0.5835}
\newcommand{\MetricQSixAuc}{0.5495}
\newcommand{\MetricQSixFOne}{0.2475}
\newcommand{\MetricQSixModel}{XGBoost}
\newcommand{\MetricRuntimeProfile}{balanced}
}{}

% Fallback defaults (overridden by auto-generated metrics when available)
\providecommand{\MetricRuntimeProfile}{N/A}
\providecommand{\MetricQSixModel}{N/A}
\providecommand{\MetricQSixAccuracy}{N/A}
\providecommand{\MetricQSixAuc}{N/A}
\providecommand{\MetricQSixFOne}{N/A}
\providecommand{\MetricQFifteenBrier}{N/A}
\providecommand{\MetricQFifteenEce}{N/A}
\providecommand{\MetricQFifteenBestFOneThreshold}{N/A}
\providecommand{\MetricQSixteenTopFeature}{N/A}
\providecommand{\MetricQSixteenTopPsi}{N/A}
\providecommand{\MetricQSeventeenRecourseRate}{N/A}
\providecommand{\MetricQEighteenMeanAuc}{N/A}
\providecommand{\MetricQEighteenAucDecay}{N/A}
\providecommand{\MetricQNineteenCovNinety}{N/A}
\providecommand{\MetricQNineteenMaxUndercoverage}{N/A}
\providecommand{\MetricQTwentyDpGapBase}{N/A}
\providecommand{\MetricQTwentyDpGapMitigated}{N/A}
\providecommand{\MetricQTwentyPolicyPass}{N/A}

\title{A Complete IEEE-Style Capstone Report for Global Tech Talent Migration Analysis}

\author{%
\IEEEauthorblockN{UT-ECE Data Science TA Team}
\IEEEauthorblockA{Department of Electrical and Computer Engineering\\
University of Tehran\\
Spring 2025 Capstone Package}
}

\begin{document}
\maketitle

\begin{abstract}
This report documents an end-to-end graduate-level data science capstone on \texttt{GlobalTechTalent\_50k.csv}. The pipeline covers data engineering, leakage diagnostics, inference, optimization, non-linear modeling, unsupervised learning, explainable AI, and production-oriented extensions (Q15--Q20). The implementation is reproducible via a profile-aware CLI and generates report-ready metrics and artifacts automatically.
\end{abstract}

\section{Problem Statement and Scope}
The project predicts \texttt{Migration\_Status} for 50,000 technical professionals while enforcing methodological controls for leakage, drift, uncertainty, and fairness. The final package is organized as a full instructional+engineering deliverable: scripts, tests, notebooks, and bilingual reports.

\section{Dataset Diagnostics}

\subsection{Target Balance}
\begin{figure}[H]
\centering
\includegraphics[width=0.88\linewidth]{../figures/report_target_balance.png}
\caption{Class distribution for \texttt{Migration\_Status}.}
\end{figure}
\textbf{Interpretation:} The dataset is moderately imbalanced and requires threshold-aware evaluation beyond raw accuracy. \\
\textbf{Decision Impact:} Model selection and threshold policy are evaluated with AUC/F1/calibration rather than accuracy alone. \\
\textbf{Limitation/Threat:} Class prevalence can shift in deployment, so historical balance should not be treated as stationary.

\subsection{Missingness Profile}
\begin{figure}[H]
\centering
\includegraphics[width=0.90\linewidth]{../figures/report_missingness_top10.png}
\caption{Top-10 missingness rates across columns.}
\end{figure}
\textbf{Interpretation:} Missingness is concentrated in operational fields, notably visa-related attributes. \\
\textbf{Decision Impact:} Features tied to post-outcome process states are either removed or treated as leakage candidates. \\
\textbf{Limitation/Threat:} Missing-not-at-random behavior can encode policy effects and induce biased estimates.

\subsection{Correlation Structure}
\begin{figure}[H]
\centering
\includegraphics[width=0.90\linewidth]{../figures/report_numeric_correlation.png}
\caption{Correlation heatmap for key numeric predictors and target.}
\end{figure}
\textbf{Interpretation:} The target correlates with activity and prominence indicators but no single predictor is sufficient. \\
\textbf{Decision Impact:} Multivariate and non-linear models are justified by mixed signal strength and interaction effects. \\
\textbf{Limitation/Threat:} Correlation is not causal; policy-sensitive confounders remain unobserved.

\subsection{Country-Level Migration Rates}
\begin{figure}[H]
\centering
\includegraphics[width=0.90\linewidth]{../figures/report_country_migration_rate.png}
\caption{Country-level migration rate ranking under minimum support filter.}
\end{figure}
\textbf{Interpretation:} Group-level outcome rates vary materially across origin countries. \\
\textbf{Decision Impact:} Fairness slicing and post-mitigation evaluation are mandatory before policy use. \\
\textbf{Limitation/Threat:} Observed regional differences may encode visa regimes, not individual readiness.

\section{Core Questions (Q1--Q6)}

\subsection{Q1 Data Engineering and Leakage}
The SQL window-function answer is exported to \texttt{code/solutions/q1\_moving\_average.sql}. Leakage diagnostics show that \texttt{Visa\_Approval\_Date} must be excluded because it is a post-outcome process artifact.

\subsection{Q3 Optimizer Dynamics}
\begin{figure}[H]
\centering
\includegraphics[width=0.88\linewidth]{../figures/q3_ravine_optimizers.png}
\caption{SGD vs Momentum vs Adam trajectories on a ravine objective.}
\end{figure}
\textbf{Interpretation:} Momentum and Adam reduce ravine oscillation and converge faster than plain SGD. \\
\textbf{Decision Impact:} For ill-conditioned objectives, adaptive or momentum-based optimizers are preferred defaults. \\
\textbf{Limitation/Threat:} Toy ravine behavior does not fully capture non-convex deep objective geometry.

\subsection{Q4 Non-Linear Models and Complexity Control}
\begin{figure}[H]
\centering
\includegraphics[width=0.86\linewidth]{../figures/q4_svm_gamma_sweep.png}
\caption{Validation sensitivity to RBF kernel width (\(\gamma\)).}
\end{figure}
\textbf{Interpretation:} Larger \(\gamma\) increases local sensitivity and can increase variance. \\
\textbf{Decision Impact:} Overfitting regimes are controlled by reducing \(\gamma\) and cross-validating the support pattern. \\
\textbf{Limitation/Threat:} Hyperparameter effect depends on feature scaling and class overlap.

\begin{figure}[H]
\centering
\includegraphics[width=0.86\linewidth]{../figures/q4_tree_pruning_curve.png}
\caption{Cost-complexity pruning tradeoff for CART.}
\end{figure}
\textbf{Interpretation:} Increasing \(\alpha\) regularizes tree size and moves along the bias-variance frontier. \\
\textbf{Decision Impact:} Selected \(\alpha\) should maximize validation generalization, not training fit. \\
\textbf{Limitation/Threat:} Pruning curves are data-split dependent and may drift under regime changes.

\subsection{Q5 Unsupervised Structure}
\begin{figure}[H]
\centering
\includegraphics[width=0.82\linewidth]{../figures/q5_kmeans_elbow.png}
\caption{WCSS elbow curve for model order selection in K-Means.}
\end{figure}
\textbf{Interpretation:} WCSS shows diminishing returns after a moderate cluster count. \\
\textbf{Decision Impact:} Elbow-based \(K\) is treated as a complexity heuristic, then validated against interpretability utility. \\
\textbf{Limitation/Threat:} Elbow location can be ambiguous when curvature is shallow.

\subsection{Q6 Explainable Capstone Model}
Current run profile: \textbf{\MetricRuntimeProfile}. \\
Capstone model: \textbf{\MetricQSixModel}. \\
AUC: \textbf{\MetricQSixAuc}, Accuracy: \textbf{\MetricQSixAccuracy}, F1: \textbf{\MetricQSixFOne}.

\begin{figure}[H]
\centering
\includegraphics[width=0.95\linewidth]{../figures/q6_shap_force_plot.png}
\caption{Local SHAP explanation for a selected high-citation candidate.}
\end{figure}
\textbf{Interpretation:} The candidate prediction is the sum of feature-level positive and negative contributions from the base value. \\
\textbf{Decision Impact:} Decision review can target dominant negative drivers instead of relying on aggregate score alone. \\
\textbf{Limitation/Threat:} SHAP explains model behavior, not structural causality.

\begin{figure}[H]
\centering
\includegraphics[width=0.88\linewidth]{../figures/q6_shap_summary.png}
\caption{Global feature influence summary for the capstone model.}
\end{figure}
\textbf{Interpretation:} Research and activity-related predictors dominate the global attribution profile. \\
\textbf{Decision Impact:} Feature governance should prioritize data quality and policy review for top-ranked drivers. \\
\textbf{Limitation/Threat:} Global importance does not reflect all subgroup-specific interactions.

\section{Advanced Production-Oriented Block (Q15--Q20)}

\subsection{Q15 Calibration and Threshold Policy}
Brier: \textbf{\MetricQFifteenBrier}, ECE: \textbf{\MetricQFifteenEce}, Best-F1 threshold: \textbf{\MetricQFifteenBestFOneThreshold}.

\begin{figure}[H]
\centering
\includegraphics[width=0.86\linewidth]{../figures/q15_calibration_curve.png}
\caption{Calibration reliability curve for the capstone model.}
\end{figure}
\textbf{Interpretation:} The calibration curve quantifies probability reliability against empirical frequencies. \\
\textbf{Decision Impact:} Operational thresholds are selected from calibrated risk estimates rather than raw scores. \\
\textbf{Limitation/Threat:} Calibration can degrade over time and must be monitored.

\begin{figure}[H]
\centering
\includegraphics[width=0.90\linewidth]{../figures/q15_threshold_tradeoff.png}
\caption{Threshold tradeoff between F1 components and expected decision cost.}
\end{figure}
\textbf{Interpretation:} Precision, recall, and expected error cost respond differently to threshold movement. \\
\textbf{Decision Impact:} The threshold is chosen by policy utility (cost matrix), not a universal default. \\
\textbf{Limitation/Threat:} Cost assumptions are context-dependent and can change by jurisdiction.

\subsection{Q16 Drift Monitoring}
Top drift feature: \textbf{\MetricQSixteenTopFeature} with PSI \textbf{\MetricQSixteenTopPsi}.

\begin{figure}[H]
\centering
\includegraphics[width=0.90\linewidth]{../figures/q16_drift_psi_top12.png}
\caption{Feature drift ranking using PSI thresholds.}
\end{figure}
\textbf{Interpretation:} A subset of features exhibits moderate-to-high instability between reference and current windows. \\
\textbf{Decision Impact:} Retraining and feature-level alerting are triggered by PSI severity bands. \\
\textbf{Limitation/Threat:} PSI detects distribution shift, not necessarily target relationship drift.

\subsection{Q17 Counterfactual Recourse}
Recourse success rate: \textbf{\MetricQSeventeenRecourseRate}.

\begin{figure}[H]
\centering
\includegraphics[width=0.78\linewidth]{../figures/q17_recourse_median_deltas.png}
\caption{Median actionable effort needed to flip near-boundary predictions.}
\end{figure}
\textbf{Interpretation:} Actionability varies across controllable features and can be quantified. \\
\textbf{Decision Impact:} Candidate guidance can focus on the lowest-effort feasible interventions. \\
\textbf{Limitation/Threat:} Counterfactual feasibility depends on real-world constraints not encoded in data.

\subsection{Q18 Temporal Backtesting and Degradation}
Mean temporal AUC: \textbf{\MetricQEighteenMeanAuc}, AUC decay: \textbf{\MetricQEighteenAucDecay}.

\begin{figure}[H]
\centering
\includegraphics[width=0.90\linewidth]{../figures/q18_temporal_degradation.png}
\caption{Rolling temporal folds: performance decay against drift proxy.}
\end{figure}
\textbf{Interpretation:} Performance changes across sequential folds and is contrasted with drift magnitude. \\
\textbf{Decision Impact:} Time-aware validation is required before production claims on future windows. \\
\textbf{Limitation/Threat:} If a true time field is absent, fallback ordering introduces uncertainty in temporal realism.

\subsection{Q19 Uncertainty Quantification}
Coverage@90: \textbf{\MetricQNineteenCovNinety}, max under-coverage: \textbf{\MetricQNineteenMaxUndercoverage}.

\begin{figure}[H]
\centering
\includegraphics[width=0.90\linewidth]{../figures/q19_coverage_vs_alpha.png}
\caption{Nominal vs empirical coverage and interval width across confidence levels.}
\end{figure}
\textbf{Interpretation:} Split-conformal intervals expose the calibration-quality tradeoff between reliability and interval width. \\
\textbf{Decision Impact:} Policy can reject low-confidence cases and route them for human review. \\
\textbf{Limitation/Threat:} Distribution shift can break finite-sample coverage guarantees.

\subsection{Q20 Fairness Mitigation Experiment}
DP gap baseline: \textbf{\MetricQTwentyDpGapBase}; DP gap mitigated: \textbf{\MetricQTwentyDpGapMitigated}; policy pass: \textbf{\MetricQTwentyPolicyPass}.

\begin{figure}[H]
\centering
\includegraphics[width=0.84\linewidth]{../figures/q20_fairness_tradeoff.png}
\caption{Fairness-performance tradeoff from baseline to mitigated model.}
\end{figure}
\textbf{Interpretation:} Reweighing shifts the operating point on the fairness-performance plane. \\
\textbf{Decision Impact:} Deployment requires explicit policy constraints on acceptable performance drop and fairness gain. \\
\textbf{Limitation/Threat:} Single-metric fairness gains can mask harms on unmonitored subgroups.

\section{Reproducibility and Artifacts}
The implementation supports profile-based execution via
\texttt{python code/scripts/full\_solution\_pipeline.py --profile \{fast,balanced,heavy\}}.
All artifacts are generated under:\\
\texttt{code/solutions/} and \texttt{code/figures/}, including schema-v2 \texttt{run\_summary.json}, Q18--Q20 CSV outputs, and report metric exports (\texttt{latex\_metrics.json}/\texttt{.tex}).

\section{Conclusion}
The project now operates as a full graduate-capstone package: it combines mathematical rigor, engineering reproducibility, production diagnostics, and explainability/fairness accountability in a single auditable workflow.

\end{document}
