\documentclass[conference]{IEEEtran}

\usepackage{amsmath,amssymb}
\usepackage{booktabs}
\usepackage{graphicx}
\usepackage{float}
\usepackage{hyperref}
\usepackage{url}
\usepackage{array}
\usepackage{enumitem}
\usepackage{multirow}
\hypersetup{hidelinks}

% =========================
% Auto-import generated metrics if available
% =========================
\IfFileExists{../solutions/latex_metrics.tex}{% Auto-generated by report_metrics_export.py
% metric_export_version=2.0
\newcommand{\MetricDataSplitStrategy}{{'base\_supervised': 'stratified\_train\_test\_80\_20', 'q18\_temporal': 'fallback\_userid\_proxy', 'q19\_uncertainty': 'train/calibration/test = 60/20/20', 'q20\_fairness': 'stratified\_train\_test\_80\_20'}}
\newcommand{\MetricExportVersion}{2.0}
\newcommand{\MetricQFifteenBestFOneThreshold}{0.2500}
\newcommand{\MetricQFifteenBrier}{0.2436}
\newcommand{\MetricQFifteenEce}{0.0327}
\newcommand{\MetricQSixteenTopFeature}{Visa\_Approval\_Date}
\newcommand{\MetricQSixteenTopPsi}{0.0013}
\newcommand{\MetricQSeventeenRecourseRate}{1.0000}
\newcommand{\MetricQEighteenAucDecay}{0.0478}
\newcommand{\MetricQEighteenMeanAuc}{0.5428}
\newcommand{\MetricQNineteenCovNinety}{0.9000}
\newcommand{\MetricQNineteenMaxUndercoverage}{0.0050}
\newcommand{\MetricQTwentyDpGapBase}{0.1550}
\newcommand{\MetricQTwentyDpGapMitigated}{0.0978}
\newcommand{\MetricQTwentyPolicyPass}{true}
\newcommand{\MetricQSixAccuracy}{0.5835}
\newcommand{\MetricQSixAuc}{0.5495}
\newcommand{\MetricQSixFOne}{0.2475}
\newcommand{\MetricQSixModel}{XGBoost}
\newcommand{\MetricRuntimeProfile}{balanced}
}{}

% =========================
% Fallback defaults (overridden by auto-generated metrics when available)
% =========================
\providecommand{\MetricRuntimeProfile}{N/A}
\providecommand{\MetricQSixModel}{N/A}
\providecommand{\MetricQSixAccuracy}{N/A}
\providecommand{\MetricQSixAuc}{N/A}
\providecommand{\MetricQSixFOne}{N/A}
\providecommand{\MetricQFifteenBrier}{N/A}
\providecommand{\MetricQFifteenEce}{N/A}
\providecommand{\MetricQFifteenBestFOneThreshold}{N/A}
\providecommand{\MetricQSixteenTopFeature}{N/A}
\providecommand{\MetricQSixteenTopPsi}{N/A}
\providecommand{\MetricQSeventeenRecourseRate}{N/A}
\providecommand{\MetricQEighteenMeanAuc}{N/A}
\providecommand{\MetricQEighteenAucDecay}{N/A}
\providecommand{\MetricQNineteenCovNinety}{N/A}
\providecommand{\MetricQNineteenMaxUndercoverage}{N/A}
\providecommand{\MetricQTwentyDpGapBase}{N/A}
\providecommand{\MetricQTwentyDpGapMitigated}{N/A}
\providecommand{\MetricQTwentyPolicyPass}{N/A}

\title{A Complete IEEE-Style Capstone Report for Global Tech Talent Migration Analysis}

\author{%
\IEEEauthorblockN{UT-ECE Data Science TA Team}
\IEEEauthorblockA{Department of Electrical and Computer Engineering\\
University of Tehran\\
Spring 2025 Capstone Package}
}

\begin{document}
\maketitle

\begin{abstract}
This report presents an end-to-end graduate-level data science capstone on \texttt{GlobalTechTalent\_50k.csv}, targeting prediction of \texttt{Migration\_Status}. The workflow integrates data engineering, leakage diagnostics, statistical inference, optimization, non-linear modeling, unsupervised learning, explainability, and production-grade extensions (Q15--Q20). The implementation is reproducible via a profile-aware command-line pipeline and exports publication-ready metrics and figures automatically.
\end{abstract}

\begin{IEEEkeywords}
Data Science, Migration Prediction, Reproducibility, Calibration, Drift Detection, Counterfactual Recourse, Fairness Mitigation, SHAP
\end{IEEEkeywords}

% =========================================================
\section{Problem Statement and Scope}
The objective is to estimate migration propensity for 50{,}000 technical professionals while enforcing methodological safeguards against leakage, drift, uncertainty, and subgroup harm. The deliverable is both instructional and production-oriented: scripts, tests, notebooks, figures, and IEEE-ready reports.

\subsection{Primary Research Questions}
\begin{itemize}[leftmargin=1.4em]
\item Can a leakage-safe supervised pipeline provide reliable migration risk estimates?
\item Are predictions calibrated enough for threshold-based policy use?
\item How robust is model quality under drift and temporal shifts?
\item Can fairness improve under explicit policy constraints with acceptable utility loss?
\end{itemize}

% =========================================================
\section{Data Diagnostics and EDA}

\subsection{Target Balance}
\begin{figure}[H]
\centering
\includegraphics[width=0.88\linewidth]{../figures/report_target_balance.png}
\caption{Class distribution for \texttt{Migration\_Status}.}
\label{fig:target_balance}
\end{figure}
\textbf{Interpretation:} The target is moderately imbalanced.\\
\textbf{Decision Impact:} Accuracy alone is insufficient; AUC/F1/calibration are primary.\\
\textbf{Threat:} Class prevalence may shift post-deployment.

\subsection{Missingness Profile}
\begin{figure}[H]
\centering
\includegraphics[width=0.90\linewidth]{../figures/report_missingness_top10.png}
\caption{Top-10 missingness rates across columns.}
\label{fig:missingness}
\end{figure}
\textbf{Interpretation:} Missingness clusters around operational and visa-related fields.\\
\textbf{Decision Impact:} Post-outcome process features are treated as leakage candidates.\\
\textbf{Threat:} Missing-not-at-random mechanisms can bias estimates.

\subsection{Correlation Structure}
\begin{figure}[H]
\centering
\includegraphics[width=0.90\linewidth]{../figures/report_numeric_correlation.png}
\caption{Correlation heatmap for key numeric predictors and target.}
\label{fig:corr}
\end{figure}
\textbf{Interpretation:} Multiple weak-to-moderate signals appear; no single dominant predictor.\\
\textbf{Decision Impact:} Multivariate and non-linear modeling is justified.\\
\textbf{Threat:} Correlation is not causation.

\subsection{Country-Level Outcome Variation}
\begin{figure}[H]
\centering
\includegraphics[width=0.90\linewidth]{../figures/report_country_migration_rate.png}
\caption{Country-level migration rates with minimum support filter.}
\label{fig:country_rate}
\end{figure}
\textbf{Interpretation:} Group-level outcome disparities are material.\\
\textbf{Decision Impact:} Fairness slicing is mandatory before policy use.\\
\textbf{Threat:} Country effects may encode policy regimes rather than individual readiness.

% =========================================================
\section{Core Questions (Q1--Q6)}

\subsection{Q1: Data Engineering and Leakage Control}
The SQL window-function solution is provided in \texttt{code/solutions/q1\_moving\_average.sql}.  
Leakage diagnosis flags \texttt{Visa\_Approval\_Date} as a post-outcome artifact and therefore excluded from training.

\textbf{Governance rule:} a feature is eligible only if timestamped at or before prediction time.

\subsection{Q3: Optimizer Dynamics on Ravine Geometry}
\begin{figure}[H]
\centering
\includegraphics[width=0.88\linewidth]{../figures/q3_ravine_optimizers.png}
\caption{SGD vs Momentum vs Adam trajectories on a ravine objective.}
\label{fig:q3_optim}
\end{figure}
\textbf{Interpretation:} Momentum/Adam reduce oscillation and speed convergence compared with plain SGD.\\
\textbf{Decision Impact:} For ill-conditioned surfaces, adaptive or momentum-based optimizers are preferred.\\
\textbf{Threat:} Toy ravine outcomes may not fully transfer to deep non-convex settings.

\subsection{Q4: Non-Linear Models and Complexity Control}
\begin{figure}[H]
\centering
\includegraphics[width=0.86\linewidth]{../figures/q4_svm_gamma_sweep.png}
\caption{Validation sensitivity to RBF width (\(\gamma\)).}
\label{fig:q4_svm}
\end{figure}
\textbf{Interpretation:} High \(\gamma\) increases local sensitivity and variance risk.\\
\textbf{Decision Impact:} Overfit control via \(\gamma\downarrow\), \(C\)-tuning, and CV.

\begin{figure}[H]
\centering
\includegraphics[width=0.86\linewidth]{../figures/q4_tree_pruning_curve.png}
\caption{Cost-complexity pruning curve for CART.}
\label{fig:q4_prune}
\end{figure}
\textbf{Interpretation:} Increasing \(\alpha\) shrinks tree complexity along the bias--variance frontier.\\
\textbf{Decision Impact:} Select \(\alpha\) by validation generalization, not training fit.

\subsection{Q5: Unsupervised Structure Discovery}
\begin{figure}[H]
\centering
\includegraphics[width=0.82\linewidth]{../figures/q5_kmeans_elbow.png}
\caption{WCSS elbow curve for K-Means model order selection.}
\label{fig:q5_elbow}
\end{figure}
\textbf{Interpretation:} Diminishing WCSS gains after moderate \(K\).\\
\textbf{Decision Impact:} \(K\) selected as complexity--utility compromise, then validated with interpretability.

\subsection{Q6: Explainable Capstone Model}
Runtime profile: \textbf{\MetricRuntimeProfile}.  
Capstone model: \textbf{\MetricQSixModel}.  
AUC: \textbf{\MetricQSixAuc}, Accuracy: \textbf{\MetricQSixAccuracy}, F1: \textbf{\MetricQSixFOne}.

\begin{figure}[H]
\centering
\includegraphics[width=0.95\linewidth]{../figures/q6_shap_force_plot.png}
\caption{Local SHAP explanation for a selected high-citation candidate.}
\label{fig:q6_local}
\end{figure}
\textbf{Interpretation:} Instance-level score equals base value plus feature contributions.\\
\textbf{Decision Impact:} Review focuses on dominant negative drivers, not only aggregate score.\\
\textbf{Threat:} SHAP explains model behavior, not causal mechanisms.

\begin{figure}[H]
\centering
\includegraphics[width=0.88\linewidth]{../figures/q6_shap_summary.png}
\caption{Global SHAP summary for the capstone model.}
\label{fig:q6_global}
\end{figure}
\textbf{Interpretation:} Research/activity indicators dominate attribution globally.\\
\textbf{Decision Impact:} Top drivers require stricter data-quality and policy oversight.\\
\textbf{Threat:} Global importance can hide subgroup interaction heterogeneity.

% =========================================================
\section{Advanced Production-Oriented Block (Q15--Q20)}

\subsection{Q15: Calibration and Threshold Policy}
Brier: \textbf{\MetricQFifteenBrier}; ECE: \textbf{\MetricQFifteenEce}; Best-F1 threshold: \textbf{\MetricQFifteenBestFOneThreshold}.

\begin{figure}[H]
\centering
\includegraphics[width=0.86\linewidth]{../figures/q15_calibration_curve.png}
\caption{Reliability (calibration) curve of the capstone model.}
\label{fig:q15_cal}
\end{figure}
\textbf{Interpretation:} Reliability gap between predicted and observed frequencies is quantified.\\
\textbf{Decision Impact:} Thresholds are selected from calibrated risk, not raw margins.\\
\textbf{Threat:} Calibration degrades under shift; requires periodic recalibration.

\begin{figure}[H]
\centering
\includegraphics[width=0.90\linewidth]{../figures/q15_threshold_tradeoff.png}
\caption{Threshold tradeoff: precision/recall/F1 and expected decision cost.}
\label{fig:q15_thr}
\end{figure}
\textbf{Interpretation:} Policy utility changes nonlinearly with threshold.\\
\textbf{Decision Impact:} Operating point is cost-matrix driven (FN vs FP asymmetry).

\subsection{Q16: Drift Detection and Monitoring}
Top drift feature: \textbf{\MetricQSixteenTopFeature}; PSI: \textbf{\MetricQSixteenTopPsi}.

\begin{figure}[H]
\centering
\includegraphics[width=0.90\linewidth]{../figures/q16_drift_psi_top12.png}
\caption{PSI-based drift ranking.}
\label{fig:q16_drift}
\end{figure}
\textbf{Interpretation:} Several predictors show moderate/high instability.\\
\textbf{Decision Impact:} Severity bands trigger alerts, retraining checks, and model revalidation.\\
\textbf{Threat:} Covariate drift does not necessarily imply concept drift.

\subsection{Q17: Counterfactual Recourse}
Recourse success rate: \textbf{\MetricQSeventeenRecourseRate}.

\begin{figure}[H]
\centering
\includegraphics[width=0.78\linewidth]{../figures/q17_recourse_median_deltas.png}
\caption{Median actionable effort to flip near-boundary negative predictions.}
\label{fig:q17_rec}
\end{figure}
\textbf{Interpretation:} Recourse burden varies by controllable feature.\\
\textbf{Decision Impact:} Guidance can prioritize low-effort, feasible interventions.\\
\textbf{Threat:} Real-world feasibility constraints may be partially unobserved.

\subsection{Q18: Temporal Backtesting and Degradation}
Mean temporal AUC: \textbf{\MetricQEighteenMeanAuc}; AUC decay: \textbf{\MetricQEighteenAucDecay}.

\begin{figure}[H]
\centering
\includegraphics[width=0.90\linewidth]{../figures/q18_temporal_degradation.png}
\caption{Rolling temporal fold performance versus drift proxy.}
\label{fig:q18_temp}
\end{figure}
\textbf{Interpretation:} Sequential performance degradation is measurable and drift-aware.\\
\textbf{Decision Impact:} Time-aware validation is required before future-window claims.\\
\textbf{Threat:} If true timestamps are absent, fallback ordering adds uncertainty.

\subsection{Q19: Uncertainty Quantification}
Coverage@90: \textbf{\MetricQNineteenCovNinety}; Max under-coverage: \textbf{\MetricQNineteenMaxUndercoverage}.

\begin{figure}[H]
\centering
\includegraphics[width=0.90\linewidth]{../figures/q19_coverage_vs_alpha.png}
\caption{Nominal vs empirical coverage and interval width.}
\label{fig:q19_uq}
\end{figure}
\textbf{Interpretation:} Conformal coverage reflects reliability-width tradeoff.\\
\textbf{Decision Impact:} Low-confidence cases can be deferred to human review.\\
\textbf{Threat:} Distribution shift can weaken finite-sample guarantees.

\subsection{Q20: Fairness Mitigation}
DP gap (baseline): \textbf{\MetricQTwentyDpGapBase}; DP gap (mitigated): \textbf{\MetricQTwentyDpGapMitigated}; Policy pass: \textbf{\MetricQTwentyPolicyPass}.

\begin{figure}[H]
\centering
\includegraphics[width=0.84\linewidth]{../figures/q20_fairness_tradeoff.png}
\caption{Fairness--performance tradeoff after mitigation (e.g., reweighing).}
\label{fig:q20_fair}
\end{figure}
\textbf{Interpretation:} Mitigation shifts operating point on utility-fairness plane.\\
\textbf{Decision Impact:} Deployment depends on explicit policy bounds for utility loss.\\
\textbf{Threat:} Improvement in one fairness metric may mask harms elsewhere.

% =========================================================
\section{Consolidated Metric Snapshot}
\begin{table}[H]
\centering
\caption{Auto-exported key outcomes}
\begin{tabular}{>{\raggedright\arraybackslash}p{0.49\linewidth} >{\raggedleft\arraybackslash}p{0.42\linewidth}}
\toprule
\textbf{Metric} & \textbf{Value}\\
\midrule
Run profile & \MetricRuntimeProfile\\
Capstone model & \MetricQSixModel\\
Q6 AUC / Accuracy / F1 & \MetricQSixAuc\ /\ \MetricQSixAccuracy\ /\ \MetricQSixFOne\\
Q15 Brier / ECE & \MetricQFifteenBrier\ /\ \MetricQFifteenEce\\
Q15 Best-F1 threshold & \MetricQFifteenBestFOneThreshold\\
Q16 Top PSI feature (value) & \MetricQSixteenTopFeature\ (\MetricQSixteenTopPsi)\\
Q17 Recourse success rate & \MetricQSeventeenRecourseRate\\
Q18 Mean AUC / AUC decay & \MetricQEighteenMeanAuc\ /\ \MetricQEighteenAucDecay\\
Q19 Coverage@90 / Max under-coverage & \MetricQNineteenCovNinety\ /\ \MetricQNineteenMaxUndercoverage\\
Q20 DP gap baseline $\rightarrow$ mitigated & \MetricQTwentyDpGapBase\ $\rightarrow$\ \MetricQTwentyDpGapMitigated\\
Q20 Policy pass & \MetricQTwentyPolicyPass\\
\bottomrule
\end{tabular}
\label{tab:metrics}
\end{table}

% =========================================================
\section{Reproducibility and Artifacts}
The full run is profile-driven:
\begin{quote}
\texttt{python code/scripts/full\_solution\_pipeline.py --profile \{fast,balanced,heavy\}}
\end{quote}

Generated outputs include:
\begin{itemize}[leftmargin=1.4em]
\item figures under \texttt{code/figures/} and \texttt{../figures/}
\item solution tables under \texttt{code/solutions/}
\item machine-readable summaries: \texttt{run\_summary.json}
\item report-ready exports: \texttt{latex\_metrics.json}, \texttt{latex\_metrics.tex}
\item Q18--Q20 CSV artifacts for auditability and grading
\end{itemize}

\section{Conclusion}
This capstone delivers a complete, auditable workflow that connects methodological rigor with production-readiness: leakage-safe modeling, calibrated decision policy, drift surveillance, uncertainty-aware routing, and fairness-constrained mitigation. The package is suitable for graduate instruction and near-production experimentation under explicit governance controls.

\end{document}
